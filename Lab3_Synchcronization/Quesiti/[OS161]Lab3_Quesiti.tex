\documentclass[12pt]{article}
\usepackage{graphicx} % Required for inserting images
\usepackage[T1]{fontenc}
\usepackage[latin1]{inputenc}
\usepackage{glossaries}
\usepackage{graphicx}
\usepackage{amsfonts}
\usepackage{amssymb}
\usepackage{listings}
\usepackage{verbatim}
\usepackage{tikz}
\usetikzlibrary{shapes,arrows}
\usepackage{tikz}
\tikzstyle{mybox} = [draw=black, thin, rectangle, rounded corners, inner ysep=5pt, inner xsep=5pt, fill=blue!15]
\newtheorem{theorem}{Teorema}
\usepackage[a4paper, top=1cm , bottom=2cm , right=2cm , left=2cm ]{geometry}

\title{\vspace{-1cm}
\textbf{[OS Internals] LAB 2.2 - QUESITI}\\ 
\vspace{0.3cm}
\small{\textit{Esempi domande d'esame}}}
\author{}
\date{}

\begin{document}
    \maketitle

    \section*{Quesito \#1}
    {\color{blue}
    Descrivere tramite pseudocodice l'implementazione di \texttt{ram\_stealmem(npages)} in OS/161 e motivare perch\'e questa effettua allocazione di memoria contigua. 
    }\\
    
    \noindent
    Una pseudo-implementazione di \texttt{ram\_stealmem(npages)} \`e la seguente: 
    \begin{verbatim}
        function ram_stealmem(npages): ptr addr
            size = npages * DIM_SINGOLA_PAGINA;
            if (primo_ind_libero+size>last_available_address)
                return 0;
            addr=primo_ind_libero; 
            primo_ind_libero=addr+size;
            return addr; 
        end
    \end{verbatim}
    L'allocazione di memoria che risulta dall'utilizzo (lato utente e lato kernel) di tale funzione \`e contigua perch\'e ad ogni chiamata di \texttt{ram\_stealmem()} il puntatore alla memoria disponibile (primo indirizzo utilizzabile) viene aumentato di una dimensione pari allo spazio richiesto per \texttt{npages}, in questo modo, finita la porzione di memoria messa a disposizione per un processo, comincia quella che occuper\`a (potenzialmente) un altro processo.\\
    Inoltre con il solo utilizzo di \texttt{ram\_stealmem()} - senza gestire quindi il rilascio dello spazio allocato - il SO proceder\`a ad allocare finch\`e c'\`e spazio disponibile fino a lanciare un'eccezione di \texttt{Out of memory}.    

    \section*{Quesito \#2}
    {\color{blue}
    Motivare l'utilizzo e la differenza tra indirizzi fisici e virtuali nella gestione della memoria di OS/161. Fornire almeno un esempio per entrambi i casi.
    }
    

    \section*{Quesito \#3}
    {\color{blue}
     Descrivere il flow e le chiamate a funzione necessarie a liberare l'address space di un processo X che termina e allocare un nuovo address space di un processo Y
    }

    \begin{enumerate}
        \item as\_destroy $\to$ free\_ppages()
        \item 
    \end{enumerate}
   

    
    
\end{document}