\documentclass[12pt]{article}
\input{packages}

\title{\vspace{-1cm}
\textbf{[OS Internals] LAB 2.1 - QUESITI}\\ 
\vspace{0.3cm}
\small{\textit{Esempi domande d'esame}}}
\author{}
\date{}

\begin{document}

\maketitle

\vspace{-2cm}
\section{Quesito \#1 {\small\color{red} (serve OS161 - Memory???)}}
{\color{blue}
Si ha la seguente implementazione di \texttt{sys\_write()}:
\begin{verbatim}
    int sys_write(int fd, userptr_t buf_ptr, size_t size) {
        int i;
        char *p = (char *)buf_ptr;
        if (fd!=STDOUT_FILENO && fd!=STDERR_FILENO) {
            kprintf("sys_write supported only to stdout\n");
            return -1;
        }
        for (i=0; i<(int)size; i++)
            putch(p[i]);
        return (int)size;
    }
\end{verbatim}

\noindent
Supponendo che standard input, output ed error siano mappati alla console (quindi non \`e
possibile nessun loro reindirizzamento a file), si dica se \`e possibile sostituire la gestione di
stdout/stderr mediante il ciclo for con uno o pi\`u dei frammenti di codice. Per ogni alternativa
rispondere s\`i/no e fornire una motivazione.

}

\section{Quesito \#2}
{
\color{blue}{
Si supponga un sistema OS/161. Descrivere la struttura \textbf{trapframe} e il suo utilizzo durante le
chiamate di sistema. In particolare, indicare la funzione del registro \texttt{v0} dell'architettura MIPS.
}
}

\noindent
In OS161, la \textbf{trapframe} \`e una struttura (\textbf{\texttt{struct}} del C) i cui campi sono associati ai registri dell'architettura del microprocessore MIPS; viene utilizzata per salvare le informazioni dei registri della CPU nel momento in cui nel sistema si verifichi un'\textbf{eccezione}.  Essa \`e utilizzata, tra le altre cose, anche durante le chiamate di sistema (il microprocessore solleva un'eccezione verso il SO) e contiene - in questo caso - le informazioni di supporto utili alla loro gestione. La trapframe viene utilizzata nello specifico dal \textbf{dispatcher} delle system call integrato nel kernel che prende in esame: 
\begin{enumerate}
    \item Il campo \texttt{tf->v0}, per contenere (prima) il codice della syscall (dal $\mu$P al kernel) e dopo l'esecuzione della syscall (dal kernel al $\mu$P) il \textbf{valore di ritorno}.
    \item I campi da \texttt{tf->a0} a \texttt{tf->a2} (associati ai rispettivi registri \texttt{a0-a2}) contengono gli \textbf{argomenti} della chiamata di sistema; 
    \item Il campo \texttt{tf->a3}, associato al relativo registro MIPS, contiene alla fine dell'esecuzione della syscall, eventuali condizione di errore.
\end{enumerate} 

\section{Quesito \#3}
{\color{blue}
    Si supponga un sistema OS/161. \`E necessario l'utilizzo di una chiamata di sistema per la
scrittura su stdin (es. sys\_write) per un processo utente? \`E  possibile utilizzare \texttt{kprintf()}
invece? Motivare le risposte.
}\\

\noindent
In \textsc{OS161} per poter scrivere sullo stream \texttt{stdin} occorre che ci sia una system call apposita che permetta di gestire il processo utente. Come tanti altri componenti, non \`e implementata in OS161, va realizzata seguendo lo schema ad esempio della \texttt{write()} dell'ANSI/C. Una \texttt{kprintf()} non potrebbe essere impiegata in quanto \`e riservata all'utilizzo da parte del kernel.

\end{document}
