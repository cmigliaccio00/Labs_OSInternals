\documentclass[12pt]{article}
\input{packages}

\title{\vspace{-1cm}
\textbf{[OS Internals] LAB 2.2 - QUESITI}\\ 
\vspace{0.3cm}
\small{\textit{Esempi domande d'esame}}}
\author{}
\date{}

\begin{document}
\maketitle

\section*{Quesito \#1}
{\color{blue}
Descrivere tramite pseudocodice l'implementazione di \texttt{ram\_stealmem(npages)} in OS/161 e motivare perch\'e questa effettua allocazione di memoria contigua. 
}\\

\noindent
Una pseudo-implementazione di \texttt{ram\_stealmem(npages)} \`e la seguente: 
\begin{verbatim}
    function ram_stealmem(npages): ptr addr
        size = npages * DIM_SINGOLA_PAGINA;
        if (primo_ind_libero+size>last_available_address)
            return 0;
        addr=primo_ind_libero; 
        primo_ind_libero=addr+size;
        return addr; 
    end
\end{verbatim}
L'allocazione di memoria che risulta dall'utilizzo (lato utente e lato kernel) di tale funzione \`e contigua perch\'e ad ogni chiamata di \texttt{ram\_stealmem()} il puntatore alla memoria disponibile (primo indirizzo utilizzabile) viene aumentato di una dimensione pari allo spazio richiesto per \texttt{npages}, in questo modo, finita la porzione di memoria messa a disposizione per un processo, comincia quella che occuper\`a (potenzialmente) un altro processo.\\
Inoltre con il solo utilizzo di \texttt{ram\_stealmem()} - senza gestire quindi il rilascio dello spazio allocato - il SO proceder\`a ad allocare finch\`e c'\`e spazio disponibile fino a lanciare un'eccezione di \texttt{Out of memory}.    

\section*{Quesito \#2}
{\color{blue}
Motivare l'utilizzo e la differenza tra indirizzi fisici e virtuali nella gestione della memoria di OS/161. Fornire almeno un esempio per entrambi i casi.
}\\

\noindent
L'uso di indirizzi fisisi e virtuali, in OS161 come in tutti i sistemi operativi che fanno uso della memoria virtuale, vengono utilizzati affinch\'e si possa avere a disposizione uno spazio di indirizzamento che \`e molto maggiore rispetto a quello fisico. Un \textbf{indirizzo fisico} \`e associato ad una locazione di memoria 'reale', un \textbf{indirizzo virtuale} (o \textbf{logico}) \`e un riferimento simbolico che sta ad indicare invece una locazione di memoria virtuale. Il programmatore, grazie a questo tipo di indirizzi, pu\`o scrivere dei programmi che siano indipendenti dalla memoria fisica realmente presente nel sistema.
Volendo fornire degli esempi nel caso di OS161:
\begin{itemize}
    \itemsep0em
    \item \textbf{Indirizzo fisico} \`e ad esempio quello che indica l'indirizzo di base del segmento \texttt{KSEG0} del kernel, indicato con la costante \texttt{MIPS\_KSEG0}; il tipo associato a tale categoria di indirizzi \`e \texttt{paddr\_t} (sta per \textit{physical address type}). 
    \item \textbf{Indirizzo virtuale} \`e ad esempio il parametro \texttt{addr}, passato alla funzione \texttt{free\_kpages()}; il tipo associato \`e in questo caso \texttt{vaddr\_t} (sta per \textit{virtual address type}).  
\end{itemize}

\section*{Quesito \#3}
{\color{blue}
    Descrivere il flow e le chiamate a funzione necessarie a liberare l'address space di un processo X che termina e allocare un nuovo address space di un processo Y
}

\begin{enumerate}
    \item \texttt{as\_destroy} $\to$ \texttt{free\_ppages()}
    \item 
\end{enumerate}


    
    
\end{document}